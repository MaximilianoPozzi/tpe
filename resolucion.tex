\documentclass[a4paper]{article}

\usepackage[spanish]{babel} % Le indicamos a LaTeX que vamos a escribir en espa�ol.
\usepackage[latin1]{inputenc} % Permite utilizar tildes y e�es normalmente
\input{Algo1Macros}% Macros especificas para especificar problemas en AyEDI
\usepackage{caratula} % Se puede descargar en ~> https://github.com/bcardiff/dc-tex

% Aca solo vamos a poner el esqueleto del documento, pero no vamos a especificar nada.

\begin{document} % Todo lo que escribamos a partir de aca va a aparecer en el documento.

% Completar los datos de la caratula
\titulo{TPE - Agricultura con drones} 
\fecha{\today}
\materia{Algoritmos y Estructuras de Datos I}
\grupo{Grupo ?}

% Completar con cuantos integrantes quieran :)
\integrante{Apellido, Nombre1}{001/01}{email1@dominio.com}
\integrante{Apellido, Nombre2}{002/01}{email2@dominio.com}
\integrante{Apellido, Nombre3}{003/01}{email3@dominio.com}
\integrante{Apellido, Nombre4}{004/01}{email4@dominio.com}

\maketitle

\section{Tipos}

\input{tipos/tipos}

\section{Campo}

\input{tipos/campo}

\input{espec/campo}

\newpage

\section{Drone}

\input{tipos/drone}

\input{espec/drone}

\newpage

\section{Sistema}

\input{tipos/sistema}

\input{espec/sistema}

%\input{espec/ejersRecu}

\newpage

\section{Funciones Auxiliares}

\input{espec/auxiliares.tex}

\end{document} %Termin�!
